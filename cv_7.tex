%%%%%%%%%%%%%%%%%%%%%%%%%%%%%%%%%%%%%%%%%
% "ModernCV" CV and Cover Letter
% LaTeX Template
% Version 1.3 (29/10/16)
%
% This template has been downloaded from:
% http://www.LaTeXTemplates.com
%
% Original author:
% Xavier Danaux (xdanaux@gmail.com) with modifications by:
% Vel (vel@latextemplates.com)
%
% License:
% CC BY-NC-SA 3.0 (http://creativecommons.org/licenses/by-nc-sa/3.0/)
%
% Important note:
% This template requires the moderncv.cls and .sty files to be in the same 
% directory as this .tex file. These files provide the resume style and themes 
% used for structuring the document.
%
%%%%%%%%%%%%%%%%%%%%%%%%%%%%%%%%%%%%%%%%%

%----------------------------------------------------------------------------------------
%	PACKAGES AND OTHER DOCUMENT CONFIGURATIONS
%----------------------------------------------------------------------------------------

\documentclass[11pt,a4paper,sans]{moderncv} % Font sizes: 10, 11, or 12; paper sizes: a4paper, letterpaper, a5paper, legalpaper, executivepaper or landscape; font families: sans or roman

\moderncvstyle{classic} % CV theme - options include: 'casual' (default), 'classic', 'oldstyle' and 'banking'
\moderncvcolor{blue} % CV color - options include: 'blue' (default), 'orange', 'green', 'red', 'purple', 'grey' and 'black'

\usepackage{lipsum} % Used for inserting dummy 'Lorem ipsum' text into the template

\usepackage[scale=0.75]{geometry} % Reduce document margins
%\setlength{\hintscolumnwidth}{3cm} % Uncomment to change the width of the dates column
%\setlength{\makecvtitlenamewidth}{10cm} % For the 'classic' style, uncomment to adjust the width of the space allocated to your name

\usepackage{multicol}

\usepackage{fontawesome}
%\faGithub\href{https://github.com/}{name} 
%----------------------------------------------------------------------------------------
%	NAME AND CONTACT INFORMATION SECTION
%----------------------------------------------------------------------------------------

\firstname{Federico} % Your first name
\familyname{Nardi} % Your last name

% All information in this block is optional, comment out any lines you don't need
\title{Curriculum Vitae}
%\address{123 Broadway}{City, State 12345}
%\mobile{(000) 111 1111}
%\phone{(000) 111 1112}
%\fax{(000) 111 1113}
\email{fenardi87@gmail.com}
%\homepage{staff.org.edu/~jsmith}{staff.org.edu/$\sim$jsmith} % The first argument is the url for the clickable link, the second argument is the url displayed in the template - this allows special characters to be displayed such as the tilde in this example
%\extrainfo{additional information}
%\photo[70pt][0.4pt]{pictures/picture} % The first bracket is the picture height, the second is the thickness of the frame around the picture (0pt for no frame)
%\quote{"A witty and playful quotation" - John Smith}
\extrainfo{\faGithub\href{https://github.com/}{schizzz8}}% <===============
%----------------------------------------------------------------------------------------

\begin{document}

%----------------------------------------------------------------------------------------
%	COVER LETTER
%----------------------------------------------------------------------------------------

% To remove the cover letter, comment out this entire block

%\clearpage
%
%\recipient{HR Department}{Corporation\\123 Pleasant Lane\\12345 City, State} % Letter recipient
%\date{\today} % Letter date
%\opening{Dear Sir or Madam,} % Opening greeting
%\closing{Sincerely yours,} % Closing phrase
%\enclosure[Attached]{curriculum vit\ae{}} % List of enclosed documents
%
%\makelettertitle % Print letter title
%
%\lipsum[1-2] % Dummy text
%\lipsum[4] % Dummy text
%
%\makeletterclosing % Print letter signature
%
%\newpage

%----------------------------------------------------------------------------------------
%	CURRICULUM VITAE
%----------------------------------------------------------------------------------------

\makecvtitle % Print the CV title

%----------------------------------------------------------------------------------------
%	EDUCATION SECTION
%----------------------------------------------------------------------------------------

\section{Education and Work Experience}

\cventry{2019--now}{Senior Autonomous Navigation Engineer}{\textsc{Pal Robotics}}{Barcelona}{}{Role: C++ Developer of Navigation Algorithms (Visual Navigation, 2D Localization, Odometry Calibration). Project Technical Lead of Robots for Intra-Logistics.}

\cventry{2014--2019}{Ph.D. in Engineering in Computer Science}{\textsc{Sapienza University of Rome}}{}{\newline Research topics: Surface Reconstruction, SLAM, Semantic Mapping}{Supervisor: Prof Giorgio Grisetti.} 

\cventry{2011--2014}{Master of Science in Artificial Intelligence and Robotics}{\textsc{Sapienza University of Rome}}{}{109/110}{Thesis: Evaluation of the most suitable representation for geometric differential operators \\ Advisor: Prof Fiora Pirri.}

\cventry{2006--2011}{Bachelor's Degree in Computer Engineering}{\textsc{University of Naples Federico II}}{}{107/110}{Thesis: Ricostruzione tridimensionale di scene e oggetti raster \\ Advisor: Prof Antonio Picariello.}

\cventry{2011}{Work Intern}{\textsc{Associazione Culturale Campi Flegrei}}{Naples}{}{Position: Tutor of Computer Science Fundamentals for motivated adult people, Computer Technician
	and Informatic Consultant.}

\cventry{2008}{Research Intern}{\textsc{Institut f{\"u}r Technische Informatik}}{St{\"u}ttgart}{}{Topics: C++ practice for electronic devices simulation, Linux fundamentals and Bash scripting.}



\section{PhD Thesis}

\cvitem{Title}{\emph{High-Level Environment Representations for Mobile Robots.}}
\cvitem{Supervisors}{Prof Giorgio Grisetti \& Prof Daniele Nardi.}
\cvitem{Description}{The thesis focuses on the problem of building high-level representations of the environment that allow mobile robots to autonomously complete complex tasks.}

%----------------------------------------------------------------------------------------
%	WORK EXPERIENCE SECTION
%----------------------------------------------------------------------------------------

\section{Skills and Practical Experience}

\subsection{Computing and Robotics}
\cvitem{}{
\begin{itemize}
	\item Professional knowledge of programming languages: C, C++, Matlab/Octave, Java, Python, Bash.
	\item Professional knowledge of Robotics libraries and tools: ROS, Eigen, OpenCV, PCL, Gazebo, OpenGL, CUDA.
	\item Daily use of Control Version Software: Git.
	\item Proved experience with IDEs and productivity applications: QtCreator, Netbeans, \LaTeX, Microsoft Office, LibreOffice.
\end{itemize}
}

\subsection{University Projects}
\cvitem{Computer Graphics}{Development of an OpneGL+GLUT app to animate a digital hand through the P5-Glove.}
\cvitem{Machine Learning}{C++ integration of a 3D Object Recognition method based on Correspondence Grouping with the 3D Generalized Hough Transform.}
\cvitem{Autonomous \& Mobile Robotics}{Implementation in ROS of a Velocity Estimation method based on the Continuous Homography for UAVs.}
\cvitem{Computer Vision}{C++ implementation of the Level-Set method for Implicit Surface Reconstruction.}

\subsection{International Events}

\cventry{2019}{International Conference on Robotics And Automation}{IEEE RAS}{Montreal}{}{Description: Poster Presentation.}

\cventry{2017}{European Robotic League Service Robots (ERL)}{Rockin@Home}{Peccioli (PI)}{}{Task: 2D Navigation.}

\subsection{International Projects}

\cventry{2015}{European Project, TRADR}{Long-term human-robot teaming for robot assisted
	disaster response}{EU FP7 ICT 609763}{}{Task: Surface Reconstruction.}

\subsection{Summer Schools}

\cventry{2015}{TRADR Summer School on Autonomous Micro Aerial Vehicles}{Fraunhofer
	IAIS}{Bonn}{}{Topics: Autonomous Navigation for Drones.}

%----------------------------------------------------------------------------------------
%	LANGUAGES SECTION
%----------------------------------------------------------------------------------------

\section{Languages}

\cvitemwithcomment{Italian}{Mothertongue}{}
\cvitemwithcomment{English}{C1}{}
\cvitemwithcomment{Spanish}{B2}{}

\section{Publications}

\cvitem{[2019]}{Irvin Aloise, Bartolomeo Della Corte, \underline{Federico Nardi} and Giorgio Grisetti. "Systematic Handling of Heterogeneous Geometric Primitives in Graph-SLAM Optimization". \textit{IEEE Robotics and Automation Letters} (RA- L).}

\cvitem{[2019]}{\underline{Federico Nardi}, Bartolomeo Della Corte and Giorgio Grisetti. "Unified Representation and Registration of Heterogeneous Sets of Geometric Primitives". \textit{IEEE Robotics and Automation Letters} (RA- L).}

\cvitem{[2018]}{\underline{Federico Nardi}, Maria T Lazaro, Luca Iocchi and Giorgio Grisetti. "Generation of laser-quality 2D navigation maps from RGB-D sensors". \textit{RoboCup Symposium}.}

\cvitem{[2018]}{\underline{Federico Nardi}, Bartolomeo Della Corte and Giorgio Grisetti. "Unified Representation	of Heterogeneous Sets of Geometric Primitives". \textit{International Conference on Robotics And Automation (ICRA) Workshop on Perception, Inference, and Learning for Joint Semantic, Geometric, and Physical Understanding}.}

\cvitem{[2016]}{Mario Gianni, \underline{Federico Nardi}, Federico Ferri, Filippo Cantucci, Manuel A. Ruiz
	Garcia, Kartik Pushparaj and Fiora Pirri. "MIOM:A MIxed-Initiative Operational
	Model in Urban Search and Rescue". \textit{International Conference on Control,
		Automation and Robotics} (ICCAR).}
	
\cvitem{[2015]}{Valsamis Ntouskos, Marta Sanzari, Bruno Cafaro, \underline{Federico Nardi}, Fabrizio Natola, Fiora Pirri and Manuel Ruiz. "Component-Wise Modeling of Articulated Objects".
	\textit{International Conference on Computer Vision} (ICCV).}

\cvitem{[2015]}{Marta Sanzari, Fabrizio Natola, \underline{Federico Nardi}, Valsamis Ntouskos, Mahmoud Qudseya and Fiora Pirri. "Rigid tool affordance matching points of regard". \textit{International Conference on Intelligent Robots and Systems (IROS) Workshop on "Learning object affordances: a fundamental step to allow prediction, planning and tool use?"}.}

\section{Teaching}

\cvitem{2017}{\textbf{Lecturer}, Fourth Lucia PhD School on "Artificial Intelligence and Robotics", \textit{Instituto Superior Tecnico}, Lisbon.}

\cvitem{2016}{\textbf{Teaching Assistant}, "Artificial Intelligence I", \textit{Sapienza University of Rome}.}

\cvitem{2016}{\textbf{Tutor}, "Seminars in AI", \textit{Sapienza University of Rome}.}

\section{References}

\renewcommand{\listitemsymbol}{} % Changes the symbol used for lists

\begin{multicols}{2}
	\subsection{Prof Giorgio Grisetti} \cvitem{}{Department of Computer, Control \& \newline Management Engineering - \newline \textit{Sapienza University of Rome} \newline Via Ariosto 25 \newline 00185 Rome, Italy \newline \textit{grisetti@diag.uniroma1.it}}
	
	
	\subsection{Prof Daniele Nardi } \cvitem{}{Department of Computer, Control \& \newline Management Engineering - \newline \textit{Sapienza University of Rome} \newline Via Ariosto 25 \newline 00185 Rome, Italy \newline \textit{nardi@diag.uniroma1.it}}
\end{multicols}

\end{document}